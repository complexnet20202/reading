%Margenes, idioma y tipo de documento
\documentclass[12pt]{extreport}
\usepackage[spanish]{babel}
\usepackage[utf8]{inputenc}
\usepackage[margin = 1.5cm]{geometry}

%Algunas símbolos matemáticos (letras caligráficas)
\usepackage{amssymb}
\usepackage{amsmath}

\usepackage{lipsum}

\makeatletter
\renewcommand{\maketitle}{\bgroup\setlength{\parindent}{0pt}
	\begin{flushright}
		\@author
	\end{flushright}
	\begin{flushleft}
		\textbf{\@title}
	\end{flushleft}
	\egroup
}
\makeatother

\title{Lectura 01\\ WHO IS THE MOST IMPORTANT CHARACTER IN FROZEN? WHAT NETWORKS CAN TELL US ABOUT THE WORLD}
\author{%
	Quiroz Castañeda Edgar \\
	Soto Corderi Sandra del Mar
	 }

\pagestyle{empty} % disables page numbers

\begin{document}
	\maketitle
	\thispagestyle{empty} % single page disable page number
	
	¿Cómo podemos determinar la importancia de los personajes dentro de una 
	película sin tener que verla? Esto se puede resolver analizando la 
	estructura de las interacciones de los personajes. 
	
	Esta estructura de personas junto con las relaciones de ellas se denomina 
	\textbf{red social}. 
	
	Una utilidad de estas redes, es que permiten predecir como podría viajar el
	conocimiento.
	Tomando a Frozen como ejemplo, Anna descubre que su hechizo puede ser curado 
	con amor verdadero gracias a los trolls, quienes conoce a través de Kristoff.
	\\	

	¿Pero cómo se puede usar esta red social para determinar la importancia de 
	los personajes? Hay varias maneras.

	Por ejemplo, supongamos que la importancia de un personaje está dada por la 
	cantidad de personajes con las que interactúa. En ese caso, Anna es la 
	protagonista, seguida por Elsa y luego Kristoff.

	Este número se denomina el grado de la persona, y es una manera de medir la 
	importancia de algún personaje, aunque no es la única.
	\\

	El grado da una idea de como es un personaje respecto a sus amigos, pero se puede 
	generalizar para obtener un análisis más profundo.

	Para esto, se puede pensar que todos los personajes comienzan con un valor 
	de 1. Luego, cada personaje suma todos los valores de sus vecinos (sus amigos). 
	Después de una ronda, cada personaje tiene su grado. Luego de dos rondas, 
	cada personajes tiene información sobre sus vecinos, y sobre los vecinos de 
	sus vecinos. Estos números reciben el nombre de $\bf{eigenvector\ de\ centralidad}$.
	
	Siguiendo este proceso y normalizando a cada paso, se obtiene una idea sobre
	que tan influyente podría llegar a ser el personaje.
	
	Usando esta técnica y tomando en cuenta la frecuencia con la que los 
	personajes hablan entre sí, Anna sigue siendo la protagonista, pero ahora 
	Kristoff queda en segundo lugar y Elsa en tercero.
	\\
	
	Pero en una película, la importancia no sólo está dada por la cantidad de 
	interacciones. En Frozen, se puede decir que Elsa es la protagonista ya que 
	su magia es el conflicto principal de la historia. 
	
	¿Cómo se determina esto? Se pueden estudiar eventos, sus causas y  
	consecuencias en otros eventos. Se denomina una \textbf{red narrativa}.
	Esta red es bastante más complicada, pero se pueden hacer análisis similares
	a los anteriores. Y probablemente resulte que el eigenvalor de centralidad 
	de Elsa sea mayor al de Anna.
	\\
	
	¿Acaso esto tiene alguna relevancia más allá de Frozen?
	Pues resulta que redes similares están presentes en nuestra vida diaria.
	Un ejemplo muy claro de esto es la Internet,
	otro son las interacciones ecológicas que ocurren en la naturaleza, 
	siendo una la cadena alimenticia.
	En ambos casos, es útil saber que cuanto cambiaría la red de remover un 
	individuo, o saber cuantos pasos toma llegar de un punto a otro.
	\\
	
	Actualmente, los científicos están investigando formas de estudiar 
	situaciones más complejas, como redes dinámicas o  que modelen diversos tipos de relación al mismo tiempo. 
\end{document}