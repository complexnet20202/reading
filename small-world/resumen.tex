%Margenes, idioma y tipo de documento
\documentclass[12pt]{article}
\usepackage[spanish]{babel}
\usepackage[utf8]{inputenc}
\usepackage[margin = 1.5cm]{geometry}

% cambiando la fuente
\renewcommand{\familydefault}{\sfdefault}

\makeatletter
\renewcommand{\maketitle}{
	\bgroup\setlength{\parindent}{0pt}

	\begin{flushright}
		\@author
	\end{flushright}

	\begin{flushleft}
		\textbf{\@title}
	\end{flushleft}

	\egroup
}
\makeatother

\title{
	Lectura 10\\ 
	``Collective dynamics of `small-world' networks''
	}
\author{
	Quiroz Castañeda Edgar \\
	Soto Corderi Sandra del Mar
	}

\begin{document}
	\maketitle
	
	Las redes dinámicas se han usado para modelar diversos tipos de sistemas 
	autoorganizables. En estos modelos se suele suponer una estructura regular o
	aleatoria. Pero empíricamente, la mayoría de las redes caen en un punto 
	intermedio. 
	
	Para entender ese `punto medio', se introduce desorden en una red regular.
	Puntualmente, partiendo de un anillo de $n$ nodos donde cada nodo tiene $k$
	enlaces, se itera sobre todos los enlaces y se reconecta con probabilidad 
	$p$ a nodo aleatorio. $p = 0$ es una red regular y $p = 1$ es una red 
	aleatoria.

	Se observó como incrementando $p$ el coeficiente de clusttering $C$ 
	disminuye de forma lenta (alto en redes reguales) pero la distancia promedio
	$L$ disminuye muy rápido (bajo en redes aleatorias). Entonces, con $p$ 
	moderada, se tiene una red de alto $C$ y pequeña $L$.

	Esto se debe a la formación de atajos importantes que disminuyen mucho la 
	distancia sin romper la conexidad.
	
	Analizando la red de colaboración de actores de IMBD, la red de distribución 
	de electricidad de EU, y la red de neuronas de una lombris, se encontró que 
	todas éstas tenían en común estas dos propiedades. Por su naturaleza de
	atajos, se les nombró redes de mundo pequeño.

	Para estudiar más posibles propiedades de estas redes, se hicieron varias 
	simulaciones.

	Para epidemias, resulta que el coeficiente de contagio $r$ y el tiempo para 
	la infecció total $T$ son relativamente pequeños.

	En autómatas celulares, una estrucutra de mundo pequeño para tareas de 
	clasificación mejoró todo método conocido.

	En teoría de juegos, resulta que entre más incremente $p$, surge menos 
	cooperación.

	Nodos osiladores de fase se sincronizan casi de inmediato, de forma muy 
	similar a neuronas en el cerebro.

	Todos estos descubrimientos son con el propósito de abrir líneas de 
	investigación de redes en diferentes campos, puesto que en el momento, es 
	una idea muy nueva que aún tiene mucho por descubrir. Se conjetura que estos
	patrones se repetirán en todo tipo de redes biológicas, sociales y en 
	sistemas artificiales.
\end{document}