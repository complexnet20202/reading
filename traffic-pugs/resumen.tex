% formato y tipo de documento
\documentclass[12pt, letterpaper]{article}
\usepackage[margin = 1.5cm]{geometry}

% cambiando la fuente
\renewcommand{\familydefault}{\sfdefault}

% redefiniendo título para seguir las especificaciones del curso
\makeatletter
\renewcommand{\@maketitle}{
    \begin{flushright}
        \@author
    \end{flushright}

    \begin{flushleft}
        \textbf{\@title}
    \end{flushleft}
}
\makeatother

% datos del título
\title{
    Lectura 03\\ 
    ``La física de la congestión de tráfico''\\
    ``Transición de fase en epidemias''
}
\author{
    Soto Corderi Sandra del Mar\\
    Quiroz Castañeda Edgar \\
}

\begin{document}
    \maketitle

    \section*{Tráfico}

    El tráfico en las grandes ciudades siempre ha sido un gran problema. Medidas
    para reducirlo, como tarifas de congestión, no funcionan de manera óptima 
    por la naturaleza compleja de las decisiones humanas que hay en el tráfico.

    Usando herramientas de la física estadística, se puede modelar el tráfico 
    como un problema de $N$ cuerpos en fuerte interacción. La modelación del 
    tráfico como un fluido que transiciona de fluido a denso permite estudiar 
    las situaciones donde el tráfica aumenta o disminuye significativamente.

    Recientemente, modelando el problema como una red compleja, se
    desarrolló el método de tarifas llamado \textit{Hotspring Pricing}. Este 
    consiste en penalizar únicamente las intersecciones atascadas el lugar de 
    toda una región céntrica. Esto para incentivar que los conductores tomen 
    rutas alternas. Se simula el flujo de los vehículos usando una
    ecuación de balance por intersección y las desiciones de los conductores 
    dado el tráfico usando elasticidad.

    En las simulaciones realizadas, se muestra como este método podría reducir 
    significativamente el tráfico en Madrid.
    Esto es un gran ejemplo de como la física de sistemas complejos ayuda a
    resolver un problema importante de la vida real.


    \section*{Epidemias}

    Una epidemia es algún suceso donde hay contagio, como enfermedades, rumores,
    videos virales, etc. El éxito de una epidemia es algo complejo.

    Hay tres factores relevantes.
    La ley de los especiales, que indica que no todos los individuos 
    tiene la mismca capacidad de propagación.
    El factor de gancho, la efectividad del fenómeno para ser transmitido. 
    Y el poder del contexto, que indica que tan suceptible es el 
    ambiente.

    Hay dos modelos populares. 
    En SIS, cada individuo suceptible se infecta con
    una probabilidad $\beta$ al tener contacto con un infectado.Y cada individuo
    infectado se recupera a suceptible con una probabilidad $\mu$. En SIR, lo 
    único que cambia es que al un infectado sanar, desarrolla inmunidad al 
    virus.

    Usándolos, se pueden formular los factores
    usando redes. Los especiales son nodos de alto grado, el gancho es
    proporcinal a $\beta$ y el contexto es la estrucutura conexa de la red.
    
    Se busca encontrar los puntos en los que la epidemia pasa a
    un estado incontrolable o cuando se extingue. Es física, esto se conoce como
    transición de fase, y es medido por un parámetro de orden. Las
    discontinuidades de este parámetro marcan los puntos críticos del cambio de
    fase.

    Algo útil es suponer que todos los nodos del mismo grado son idénticos. Esto 
    se conoce como aproximar un campo medio heterogéneo.

    Teóricamente, cuando hay una cantidad infinita de nodos en un campo
    heterogéneo, el virus no desaparece. Pero en la práctica, esto no sucede.
    Hay métodos recientes para aproximar estos valores en casos reales.

    Aún así, muchos aspectos siguen siendo temas abiertos de invertigación.
\end{document}