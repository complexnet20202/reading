% formato y tipo de documento
\documentclass[12pt, letterpaper]{article}
\usepackage[margin = 1.5cm]{geometry}

% cambiando la fuente
\renewcommand{\familydefault}{\sfdefault}

% redefiniendo título para seguir las especificaciones del curso
\makeatletter
\renewcommand{\@maketitle}{
    \begin{flushright}
        \@author
    \end{flushright}

    \begin{flushleft}
        \textbf{\@title}
    \end{flushleft}
}
\makeatother

% datos del título
\title{
    Lectura 08\\ 
    ``Why don't patients get sick in sync? Modelers find statistical clues''
}
\author{
    Soto Corderi Sandra del Mar\\
    Quiroz Castañeda Edgar \\
}

\begin{document}
    \maketitle

    En 1950, el epidemólogo Philip Sartwell encontró que históricamente, las
    epidemias de tifoidea, polio, malaria y sarampión tenían en común que el 
    periodo de incubación seguía una distribución lognormal. Esto resulto 
    difícil de explicar, pues los mecanismos de infección de estas enfremedades
    no tienen mucho en común.
    
    Hace poco Jacob Scott, oncólogo en la Clínica de Cleyeland, et al. 
    desarrollaron un modelo matemático de crecimiento de tumores que resulta en
    este tipo de distribución. Si bien los mecanismo de infección varía entre
    tumores y otras enfermedades, esto muestra como las concepciones actuales
    sobre periodos de incubación no son del todo correctas.

    La idea general de este modelo es considerar a las células como una red 
    donde la adyacencia está dada por la cercanía de las células. Cuando una 
    célula muere, una célula infectada puede tomar su lugar. 

    Al inicio, la mayoría de las células muertas son sanas, por lo que la 
    infección se propaga rápido. Pero entre más células infectadas haya, menos 
    tejido por infectar hay. Como la infección se transmite por contacto, la 
    infección se ralentiza en cuanto más tiempo pasa.

    Dependiendo del sistema inmune de cada persona, la cantidad de tejido 
    infectado necesario para mostrar síntomas varía. Como la infección se va 
    ralentando, llegar a este punto crítico varía mucho entre personas, lo que 
    da lugar a la distribución lognormal.

    Steven Strogatz propuso que la distribución lognormal observada por Sartwell
    se puede deber a que las enfermedades se incuban siguiendo un modelo de 
    células sanas e infectadas.

    Pero que este mecanismo no es verdad por ejemplo para infecciones virales.
    Sin embargo Katia Koelle, ecóloga en la Universidad de Emory, hace notar que
    estos fenómenos también siguen una distribución lognormal.

    Una posible explicación, que ignora el arreglo de las células, es que si una
    enfermedad crece de forma exponencial y la población es expuesta de forma
    normal, esto resulta en una incubación lognormal.

    Este extraño caso hace notar que falta mucho por descubrir para entender 
    como se propagan e incuban enfermedades. Esto permitiría mejorar los métodos
    de contención de epidemias, entre otras cosas.

    Por su parte, Strogatz et al. publicaron recientemente en revistas de 
    biología, con el fin de llamar la atención para probar todos estos modelos 
    matemáticos con datos reales.

    \section*{Comentario}
    Resulta fascinante como fenómenos observados hace tanto sigan sin 
    explicación.

    Y como en tiempo moderno finalmente se les está dando algún tipo de 
    explicación, aunque sea parcial.

    Es muy importante continuar esta línea de investigación no sólo para 
    entender los procesos de incubación, sino para poder desarollar medidas de 
    prevención lo más efectivas posible.
\end{document}