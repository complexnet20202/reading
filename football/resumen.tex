% formato y tipo de documento
\documentclass[12pt, letterpaper]{article}
\usepackage[margin = 1.5cm]{geometry}

% cambiando la fuente
\renewcommand{\familydefault}{\sfdefault}

% redefiniendo título para seguir las especificaciones del curso
\makeatletter
\renewcommand{\@maketitle}{
    \begin{flushright}
        \@author
    \end{flushright}

    \begin{flushleft}
        \textbf{\@title}
    \end{flushleft}
}
\makeatother

% datos del título
\title{
    Lectura 14\\ 
    ``A network theory analysis of football strategies''
}
\author{
    Soto Corderi Sandra del Mar\\
    Quiroz Castañeda Edgar \\
}

\begin{document}
    \maketitle

    El análisis de pártidos de fútbol, así como estilos de juego y evaluación 
    de un equipo son cosas que por lo general no se puede averiguar con datos 
    estadísticos. Esto es debido a que las puntuaciones no reflejan muy bien el 
    desempeño del equipo. 

    Pero con una gran cantidad de datos publicados por la FIFA sobre la Copa 
    Mundial de 2010, se han desarrollado algunas técnicas basadas en redes 
    complejdas para evaluar la estrategia de un equipo.

    En este caso, se construyó una red con jugadores como nodos. Los enlaces 
    dirigidos con peso están dados por la frecuencia en la que un jugador pasa 
    el balón a otro. Se analizan varias medidas de conexidad y centralidad para 
    analizar una partida.

    Por ejemplo, una medida sencilla es la conexidad de enlaces. Esto se puede 
    ver como la cantidad de pases que necesitan ser interceptados para romper el
    flujo de un equipo.

    También se definió la distancia entre dos jugadores como la suma del 
    inverso de los pesos de las aristas. Dos jugadores son cercanos si hay 
    muchas pases, no necesariamente pases directos, entre ellos.

    Se definió el agrupamiento como la cantidad de ciclos de tres jugadores, 
    pues esto permitiría rutas alternas para un pase. Un agrupamiento alto 
    indica que no hay muchos jugadores que acaparen los pases.

    Tomando el porcentaje de secuencias que pasan por $i$, se obtiene una idea 
    de que tan importante es $i$ como intermediario de pases. Esto se conoce 
    como intermediación. En general, se busca que se tenga una intermediación 
    baja, pues indica poca dependencia de un jugador.

    Además, una red donde todos los nodos están conectados con todos se 
    conoce como un clique. El tamaño del máximo clique indica la cantidad de 
    jugadores que hacen pases a todos entre sí.

    Por último, se define la popularidad de un jugador como proporcional a la 
    cantidad de pases que recibe, pero dando más peso a los pasos de jugadores 
    más populares.

    Con estas técnicas, se analizó el desempeño de los octavos de final. Se 
    encontró que España y Holanda tuvieron una alta conexidad de enlaces y 
    alto clique máximo.

    España mostró un mayor cantidad de pases promedio (equipo muy dinámico), y 
    Holanda una menor intermediación promedio (poca dependencia en jugadores 
    específicos). En ambos casos esto refleja los estilos de ambos equipos.

    Además, en el caso de estos dos, también se notó que la popularidad de los 
    jugadores es aproximada a el ranking de importancia que le suelen dar los 
    analistas, con España teniendo una diferencia más grande de popularidad 
    entre jugadores.

    Se obtuvieron resultados igual de acertados al evaluar a los 16 equipos. Y 
    en especial al evaluar el desempeño por jugador de Alemania y Uruguay.

    Estos métodos sólo toman en cuenta pases dentro del mismo equipo a 
    posteriori. Cosas como la evolución temporal, las intercepciones de pases y
    los tipos para gol podrían tomarse en cuenta para tener un análisis más 
    detallado.

    Sin embargo, como una aproximación inicial, lo estudiado aquí resultó muy 
    acertado.

    \section*{Comentario}

    La disponibilidad de datos sobre gran cantidad de temas permite su análisis 
    muy a fondo de manera que no era posible antes.

    Con esto se pueden descubrir patrones que antes se le atribuían a la 
    intuición humana y usarse para mejorar el desempeño de gran cantidad de 
    actividades.

    Es interesante ver como todo esto se desarrolla en tan diferentes ámbitos.

\end{document}