%Margenes, idioma y tipo de documento
\documentclass[12pt]{extreport}
\usepackage[spanish]{babel}
\usepackage[utf8]{inputenc}
\usepackage[margin = 1.5cm]{geometry}


\makeatletter
\renewcommand{\maketitle}{
	\bgroup\setlength{\parindent}{0pt}

	\begin{flushright}
		\@author
	\end{flushright}

	\begin{flushleft}
		\textbf{\@title}
	\end{flushleft}

	\egroup
}
\makeatother

\title{
	Lectura 06\\ 
	``The Big Idea Behind Big Data''
	}
\author{
	Quiroz Castañeda Edgar \\
	Soto Corderi Sandra del Mar
	}

\begin{document}
	\maketitle
	
	Durante el inicio de la epidemia del H1N1, científicos en el Centro para el
	Control de las Enfermedades (CDC) predijeron exitosamente como evolucionaría
	la epidemia. Con esto, salvaron muchas vidas y previnieron potenciales 
	desastres económicos.

	Para esto modelaron la epidemia usando redes. 
	
	Este tipo de redes surge naturalmente en muchos ámbitos, como redes 
	sociales, cadenas alimenticias, redes de comercio, o interacción entre 
	genes.

	Y esto se conoce desde hace inicios del siglo pasado. Pero este tipo de 
	redes son enormes, y están altamente conectadas. Por lo que su 
	comportamiento tiene una complejidad muy alta. 	No había sido posible 
	manejar esta cantidad de información hasta que las computadoras tuvieron 
	dicha capacidad a finales del siglo pasado.

	Con este recursos de análisis, se pudieron encontrar propiedades inesperadas.
	Por ejemplo, resulta que en general, bastan seis personas para encontrar una
	trayectoria de conocidos entre dos habitantes de Estados Unidos. Es decir, 
	el diámetro de la red es aproximadamente 6. Esta red muestra un mundo 
	pequeño.

	Otra propiedad, es que la mayoría de las redes tienen una estrucutra 
	similar. Tanto el internet como el genóma humano muetras una estrucura libre
	de escala. Esto significa que es suficiente estudiar una estructura 
	abstracta para obtener muchas propiedades presentes en todas las redes.

	Estas propiedades inesperadas revelan la utiliad de una nueva perspectiva. 
	No es necesario estudiar las unidades básicas de algo para entenderlo. Hay 
	aspectos que sólo surgen cuando se tiene un sistema más complejo. Estudiar 
	algunos fenómenos desde la tería de redes permite un acercamiento desde lo
	más general. Y esto a su vez permite estudiar fenómenos tan grandes que 
	anteriormente no era factible estudiar.

	Por último, la teoría de redes permite estudiar el mundo creado por los 
	humanos. Esto podría ser peligroso si se usara con fines equivocados.

	La teoría de redes es la primer ciencia que surge de la revolución digital, 
	y ha dado herramientas fundamentales para entender nuestro mundo hoy en día.
	Entre más conocemos sobre nuestro mundo, es necesario entender las 
	implicaciones y posibles riesgos los descubrimientos.
	
\end{document}