% formato y tipo de documento
\documentclass[12pt, letterpaper]{article}
\usepackage[margin = 1.5cm]{geometry}

% cambiando la fuente
\renewcommand{\familydefault}{\sfdefault}

% redefiniendo título para seguir las especificaciones del curso
\makeatletter
\renewcommand{\@maketitle}{
    \begin{flushright}
        \@author
    \end{flushright}

    \begin{flushleft}
        \textbf{\@title}
    \end{flushleft}
}
\makeatother

% datos del título
\title{
    Lectura 04\\ 
    ``Never Underestimate the Intelligence of Trees''
}
\author{
    Soto Corderi Sandra del Mar\\
    Quiroz Castañeda Edgar \\
}

\begin{document}
    \maketitle
    Suzanne Simard, profesora en el Departamento de Bosques y Conservación en la
    Universidad de Columbia Británica, ha pasado lo últimos veinte años 
    estudiando como los árboles se comunican a través de sus raíces.

    Por medio de conexiones de hongos en sus raíces, diferentes árboles son 
    capaces de intercambiar recursos, detectar parentezco e informar acerca de
    plagas o demás peligros. Estas conexiones forman una red libre de escala, el
    mismo tipo de red que formas las neuronas en un cerebro humano.

    Este tipo de redes proveen una estrucutra robusta para interacciones 
    complejas entre los individuos.

    En este caso, un árbol reacciona a la información y a los recursos 
    recibidos, pero estas conexiones van más allá.
    En sus experimentos, Simard ha registrado que un árbol puede reconocer a sus
    crías y reaccionar diferente dependiendo de las necesidades. 
    Por ejemplo, un árbol es capaz de matar a sus crías si detecta que sus 
    condiciones no son favorables. 
    También se ha registrado como si un árbol detecta que las condiciones son 
    favorables, pero aún así hay algún árbol joven con carencia de recursos, 
    el árbol madre proveera hasta que la cría se recupere.

    Y este comportamiento no sólo es una reacción. Simard indica que analizando
    los anillos de los árboles, se obtiene una idea sobre las interacciones
    importantes históricas que ha tenido el árbol. Los árboles recuerdan y 
    actuan en base esos recuerdos.

    Más que sólo dar nuevas perspectivas para la botánica, esto lleva a un 
    replantiamiento sobre el concepto de inteligencia, que Samard opina está 
    basado únicamente en la inteligencia animal en la ciencia occidental. 
    Aunque no sea fácil de percibir, resulta que las plantas presentan un 
    comportamiento inteligente y una forma de consciencia sobre sí mismos, su 
    entorno y sus semejantes.

    Hay que ser más responsables y respetuoso con el mundo vegetal. No sólo por 
    su increíble complejidad, si no porque estas dos cosas deben estar presentes
    para mejorar la sociedad.
\end{document}