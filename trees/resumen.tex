%Margenes, idioma y tipo de documento
\documentclass[12pt]{extreport}
\usepackage[spanish]{babel}
\usepackage[utf8]{inputenc}
\usepackage[margin = 1.5cm]{geometry}


\makeatletter
\renewcommand{\maketitle}{
    \bgroup\setlength{\parindent}{0pt}

    \begin{flushright}
        \@author
    \end{flushright}

    \begin{flushleft}
        \textbf{\@title}
    \end{flushleft}

    \egroup
}
\makeatother

\title{
    Lectura 04\\ 
    ``Never Underestimate the Intelligence of Trees''
}
\author{
    Soto Corderi Sandra del Mar\\
    Quiroz Castañeda Edgar \\
}

\begin{document}
    \maketitle
    Suzanne Simard, profesora en el Departamento de Bosques y Conservación en la
    Universidad de Columbia Británica, ha pasado lo últimos veinte años 
    estudiando como los árboles se comunican a través de sus raíces.

    Por medio de conexiones de hongos en sus raíces, diferentes árboles son 
    capaces de intercambiar recursos, detectar parentezco e informar acerca de
    plagas o demás peligros. Estas conexiones forman:w
     una red libre de escala, el
    mismo tipo de red que formas las neuronas en un cerebro humano.

    ¿En qué nivel de inteligencia entra este comportamiento?
\end{document}