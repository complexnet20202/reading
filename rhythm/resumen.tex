%Margenes, idioma y tipo de documento
\documentclass[12pt]{article}
\usepackage[spanish]{babel}
\usepackage[utf8]{inputenc}
\usepackage[margin = 1.5cm]{geometry}

% cambiando la fuente
\renewcommand{\familydefault}{\sfdefault}

\makeatletter
\renewcommand{\maketitle}{
	\bgroup\setlength{\parindent}{0pt}

	\begin{flushright}
		\@author
	\end{flushright}

	\begin{flushleft}
		\textbf{\@title}
	\end{flushleft}

	\egroup
}
\makeatother

\title{
	Lectura 09\\ 
	``The Beats That Keep The Beat''
	}
\author{
	Quiroz Castañeda Edgar \\
	Soto Corderi Sandra del Mar
	}

\begin{document}
	\maketitle
	
	A veces nos encontramos con que los animales realizan acciones fuera de su naturaleza y que parecen humanas. Ejemplos de esto se encuentran en Snowball la cacatúa ue bailaba al ritmo de la música, Koko el ave con un gran vocabulario, la banda de elefantes en Tailandia, Hoover la foca que podía hablar como una persona y Ronan el león marino que podía mantener los ritmos moviendo su cabeza. 

	Edawrd Large y Joel Snyder, neurologos, observaron que ciertas señales 
	dentro de las partes del cerebro correspondientes al audio y al movimiento
	sincronizaban su frecuencia con la música en el entorno, tanto en humanos 
	como en animales no verbales.

	A raíz de estos fenómenos, surgieron dos teorías para explicar esto.

	Primero, es que el ritmo es un producto secundario de la habilidad de 
	tener algún tipo de habla. Esto es la capacidad de imitar sonidos que no son
	naturales. Como elefantes imitando camiones, o pericos imitando humanos.
	Según esta explicación, la habilidad de hablar fortaleció lo suficiente las 
	conexiones entre sensación auditiva y habilidad motriz para poder reaccionar 
	al ritmo. Todos los animales presentan esto, pero sólo en aquellos con habla
	las conexiones son lo suficientemente fuertes como para manifestarse como 
	un sentido rítmico musical.

	Otra explicación indica que el ritmo es algo innato de todo cerebro, pues 
	este es necesario para su funcionamiento. Así que la expresión rítmica 
	depende más del ambiente que de las conexiones neuronales por especie.
	Animales sociales con alta inteligencia son más propicios a imitar y
	aprender, y por lo tanto a desarrollar ritmo, pero bajo circunstancias 
	adecuadas, se ha visto como gran variedad de animales puede desarrollarlo.

	En este caso, el ritmo humano no tiene nada de especial. Es sólo que la 
	condición humana propicia el desarrollo de una habilidad presente en todos 
	los animales.
	
	\section*{Comentario}
	Lo más natural sería que todos los cerebros animales posean la estructura 
	neuronal suficiente para tener la habilidad de comprensión musical y la 
	manera en que se exprese depende del ambiente. Por ejemplo los pericos, las 
	focas y elefantes al ser animales sociales con capacidad de reproducir 
	sonidos tienen más facilidad para entender el ritmo. Mientras que otros 
	animales como los leones marinos y los monos con el entrenamiento adecuado 
	y en el ambiente correcto son capaces de hacerlo.
	
\end{document}