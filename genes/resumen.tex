% formato y tipo de documento
\documentclass[12pt, letterpaper]{article}
\usepackage[margin = 1.5cm]{geometry}

% cambiando la fuente
\renewcommand{\familydefault}{\sfdefault}

% redefiniendo título para seguir las especificaciones del curso
\makeatletter
\renewcommand{\@maketitle}{
    \begin{flushright}
        \@author
    \end{flushright}

    \begin{flushleft}
        \textbf{\@title}
    \end{flushleft}
}
\makeatother

% datos del título
\title{
    Lectura 11\\ 
    ``The human desease network''
}
\author{
    Soto Corderi Sandra del Mar\\
    Quiroz Castañeda Edgar \\
}

\begin{document}
    \maketitle
    
    Se han hecho muchos estudios que realacionan ciertos genes con ciertas 
    enfermedades, aunque la mayoría de los estudios son sobre casos 
    particulares.
    Usando toda esa información, se crea una red bipartita entre 1777 genes y 
    1284 enfermedades con base en si un gen propicia el desarrollo de cierta 
    enfermedad.

    Tomando como base esta red, se define una red de genes (DGN) y una red de 
    enfermedades (HDN) con las adyacencias dadas si tiene un vecino en común en
    la red bipartita.

    En la HDN , se notó que las enfermedades se agruparon en clusters. Los más 
    grandes, como el del cáncer, incluyen gran variedad genética, como 
    enfermedades precursoras. Pero tienen una ausencia de gran parte de las 
    enfermedades, las cuáles forman sus propios clusters.

    En ambas redes se notó que pequeño de la componente gigante. Esto muetra 
    gran clustering entre enfermedades.

    La primera hipótesis que se quiso probar con esta red es que genes en un
    mismo clustter afectan las mismas áreas funcionales. Para esto, se apoyó de 
    una red de interacción de proteínas. Para esto, se midieron sus habilidades
    funcionales, su rol en el organismo y su localización en las células de las
    proteínas relacionandos con genes en un mismo cluster. También se midió la 
    homogeneidad de los tejidos que afectaban y sus coeficientes de correlación
    de Pearson. En todos los casos, resultó que los genes en el mismo cluster sí
    afectan las mismas funciones.
    
    La segunda hipótesis que se quería corroborar era que genes involucrados con 
    enferemdades en general son genes escenciales con papales importantes, como
    en la producción de proteínas activas (hubs). Resultó que hay pocos genes 
    escenciales relacionados con enfermedades, pero estos pocos son muy activos.
    Los genes no escenciales relacionados con enfermedades suelen actuar de 
    forma más aislada y no están relacionados con proteínas hub.

    Una explicación para esto es que mutaciones en genes importantes se expresan
    fuertemente. De ser estas enferemedades, entonces el individuo tiene 
    desventaja genética, y probablemente estos rasgos desaparezcan pronto.

    Las únicas enfermedades que pueden perdurar son aquellas que afectan a los 
    individuos de forma no escencial.

    Usando datos extra, se verificó que la estructura general de la red no 
    cambia, así que probablemente estos hallazgos sean válidos para el futuro.

    \section*{Comentario}

    Me parece impresionante que se tengan ya los recursos para analizar tan a 
    fondo la relación entre genes y enfermedades.

    Las redes proporcionan el marco de trabajo perfecto para entender las 
    relaciones tan convolucionadas que hay en el funcionamiento biológico, y hay
    que aprovecharlas.

    Esto abre la puerta a muchos trabajos que podrían beneficiar a mucha gente.
\end{document}