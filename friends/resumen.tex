%Margenes, idioma y tipo de documento
\documentclass[12pt]{extreport}
\usepackage[spanish]{babel}
\usepackage[utf8]{inputenc}
\usepackage[margin = 1.5cm]{geometry}


\makeatletter
\renewcommand{\maketitle}{
	\bgroup\setlength{\parindent}{0pt}

	\begin{flushright}
		\@author
	\end{flushright}

	\begin{flushleft}
		\textbf{\@title}
	\end{flushleft}

	\egroup
}
\makeatother

\title{
	Lectura 02\\ 
	``How Network Math Can Help You Make Friends''}
\author{
	Quiroz Castañeda Edgar \\
	Soto Corderi Sandra del Mar
	}

\begin{document}
	\maketitle
	
	¿Cómo se pueden estudiar una red social de algún lugar dado? ¿Qué estructura
	tendría esta red?

	Para esto, se estudia la distribución de la cantidad de amigos de cada
	persona. Para ilustrarlo, veamos dos ejemplos extremos.

	Primero, supongamos que cada persona puede tener a lo más una cantidad fija,
	digamos cuatro, de amigos y todos quieres maximizar su cantidad de
	amigos. 

	En general,la mayor parte de las personas tendrán cuatro amigos.

	Entonces, si se observa la distribución de la cantidad de amigos, está tiene
	muchas personas con cuatro amigos y muy pocas personas con tres, dos o un
	amigo.

	En el caso opuesto, todas las amistades se forman de manera aleatoria.

	Para averiguar la distribución en este caso, supongamos que se selecciona 
	una amistad arbitraria y se quiere saber si una persona dada es parte de esa
	amistas. Hay $\frac{n(n-1)}{2}$ posibles amistades en toda la red y $n-1$ 
	posibles amistades para la persona dada. Entonces, la probabilidad de que la
	persona dada esté en la amistad arbitraria es $\frac{n-1}{\frac{n(n-1)}{2}} =
	\frac{2}{n}$.

	Esto es que toda persona tiene una probabilidad de $\frac{2}{n}$ de ser 
	amigo de cualquier otra persona.

	Es una distribución binomial. En esta, la mayoría de las personas tienen una
	cantidad promedio de amigos. Hay muy pocos casos extremos donde hay personas
	con muy pocos o con muchos amigos.
	
	¿Alguna de estos casos modela algo real? Resulta que no. La cantidad de
	amistades no tiene límite fijo como el primer modelo, y tener pocos o muchos
	amigos no es especialmente raro como en el segundo modelo.

	Se denomina a estas redes como redes de libre escala y aparecen en todo tipo 
	de redes reales. Empíricamente, estas se rigen por una regla de adjunción 
	preferencial. Esto es nuevas adyacencias prefieren nodos de grando alto.

	Intuitivamente, significa que una persona con muchos amigos puede conocer a
	más personas y hacer más amigos que una persona con pocos amigos.

	Hay muchos nodos de grado pequeño, pero también hay una cantidad
	considerable de nodos de grados altos. Representan los puntos importantes de
	tránsito de la red, como personas populares, o routers imporantes. El hecho
	de que haya varios hace este tipo de redes robustas a fallas, pues hay rutas
	alternativas para un camino necesario.

	¿Y cuál es su distribución? En la propuesta original de estas redes, se las
	describió usando una distribución de leyes de potencias $F(x) = 
	\frac{1}{x^k}$. Pero estudios recientes han observado que si bien esta
	distribución es común, muchos fenómenos con adyacencia preferencial siguen 
	más bien distribuciones exponenciales o log-normales.

	Al ser todo esto una ciencia donde los objetos de estudio surgen 
	naturalmente, la caracterización de estas redes sigue siendo un punto de 
	discución.

	\section*{Ejericios}

	\begin{enumerate}
		\item Muchos ciclos.
		\item Cuatro personas, en los vértices de un tetraedro
		\item Habría que modelarlo con una digráfica.
		\item $\frac{n(n-1)}{2}$.
	\end{enumerate}

\end{document}